\hypertarget{chapter-3-pointers-and-strings}{%
\subsection{CHAPTER 3: Pointers and
Strings}\label{chapter-3-pointers-and-strings}}

The study of strings is useful to further tie in the relationship
between pointers and arrays. It also makes it easy to illustrate how
some of the standard C string functions can be implemented. Finally it
illustrates how and when pointers can and should be passed to functions.

In C, strings are arrays of characters. This is not necessarily true in
other languages. In BASIC, Pascal, Fortran and various other languages,
a string has its own data type. But in C it does not. In C a string is
an array of characters terminated with a binary zero character (written
as \textbf{'\textbackslash0'}). To start off our discussion we will
write some code which, while preferred for illustrative purposes, you
would probably never write in an actual program. Consider, for example:

\begin{minted}{c}
    char my_string[40];

    my_string[0] = 'T';
    my_string[1] = 'e';
    my_string[2] = 'd':
    my_string[3] = '\0';
\end{minted}

While one would never build a string like this, the end result is a
string in that it is an array of characters \textbf{terminated with a
nul character}. By definition, in C, a string is an array of characters
terminated with the nul character. Be aware that "nul" is \textbf{not}
the same as "NULL". The nul refers to a zero as defined by the escape
sequence \textbf{'\textbackslash0'}. That is it occupies one byte of
memory. NULL, on the other hand, is the name of the macro used to
initialize null pointers. NULL is \#defined in a header file in your C
compiler, nul may not be \#defined at all.

Since writing the above code would be very time consuming, C permits two
alternate ways of achieving the same thing. First, one might write:

\begin{minted}{c}
    char my_string[40] = {'T', 'e', 'd', '\0',};    
\end{minted}

But this also takes more typing than is convenient. So, C permits:

\begin{minted}{c}
    char my_string[40] = "Ted";
\end{minted}

When the double quotes are used, instead of the single quotes as was
done in the previous examples, the nul character (
\textbf{'\textbackslash0}' ) is automatically appended to the end of the
string.

In all of the above cases, the same thing happens. The compiler sets
aside an contiguous block of memory 40 bytes long to hold characters and
initialized it such that the first 4 characters are
\textbf{Ted\textbackslash0}.

Now, consider the following program:

-----------  Program 3.1  -----------------------------------
\inputminted[breaklines]{c}{../src/ch3-1.c}
--------- end program 3.1 -----------------------------------


In the above we start out by defining two character arrays of 80
characters each. Since these are globally defined, they are initialized
to all \textbf{'\textbackslash0}'s first. Then, \textbf{strA} has the
first 42 characters initialized to the string in quotes.

Now, moving into the code, we declare two character pointers and show
the string on the screen. We then "point" the pointer \textbf{pA} at
\textbf{strA}. That is, by means of the assignment statement we copy the
address of \textbf{strA{[}0{]}} into our variable \textbf{pA}. We now
use \textbf{puts()} to show that which is pointed to by \textbf{pA} on
the screen. Consider here that the function prototype for
\textbf{puts()} is:

\begin{minted}{c}
    int puts(const char *s); 
\end{minted}

For the moment, ignore the \textbf{const}. The parameter passed to
\textbf{puts()} is a pointer, that is the \textbf{value} of a pointer
(since all parameters in C are passed by value), and the value of a
pointer is the address to which it points, or, simply, an address. Thus
when we write \textbf{puts(strA);} as we have seen, we are passing the
address of \textbf{strA{[}0{]}}.

Similarly, when we write \textbf{puts(pA);} we are passing the same
address, since we have set \textbf{pA = strA;}

Given that, follow the code down to the \textbf{while()} statement on
line A. Line A states:

While the character pointed to by \textbf{pA} (i.e. \textbf{*pA}) is not
a nul character (i.e. the terminating \textbf{'\textbackslash0}'), do
the following:

Line B states: copy the character pointed to by \textbf{pA} to the space
pointed to by \textbf{pB}, then increment \textbf{pA} so it points to
the next character and \textbf{pB} so it points to the next space.

When we have copied the last character, \textbf{pA} now points to the
terminating nul character and the loop ends. However, we have not copied
the nul character. And, by definition a string in C \textbf{must} be nul
terminated. So, we add the nul character with line C.

It is very educational to run this program with your debugger while
watching \textbf{strA}, \textbf{strB}, \textbf{pA} and \textbf{pB} and
single stepping through the program. It is even more educational if
instead of simply defining \textbf{strB{[}{]}} as has been done above,
initialize it also with something like:

\begin{minted}{c}
    strB[80] = "12345678901234567890123456789012345678901234567890"
\end{minted}

where the number of digits used is greater than the length of
\textbf{strA} and then repeat the single stepping procedure while
watching the above variables. Give these things a try!

Getting back to the prototype for \textbf{puts()} for a moment, the
"const" used as a parameter modifier informs the user that the function
will not modify the string pointed to by \textbf{s}, i.e. it will treat
that string as a constant.

Of course, what the above program illustrates is a simple way of copying
a string. After playing with the above until you have a good
understanding of what is happening, we can proceed to creating our own
replacement for the standard \textbf{strcpy()} that comes with C. It
might look like:

\begin{minted}{c}
   char *my_strcpy(char *destination, char *source)
   {
       char *p = destination;
       while (*source != '\0')
       {
           *p++ = *source++;
       }
       *p = '\0';
       return destination;
   }   
\end{minted}

In this case, I have followed the practice used in the standard routine
of returning a pointer to the destination.

Again, the function is designed to accept the values of two character
pointers, i.e. addresses, and thus in the previous program we could
write:

\begin{minted}{c}
    int main(void)
    {
        my_strcpy(strB, strA);
        puts(strB);
    }    
\end{minted}

I have deviated slightly from the form used in standard C which would
have the prototype:

\begin{minted}{c}
    char *my_strcpy(char *destination, const char *source);  
\end{minted}

Here the "const" modifier is used to assure the user that the function
will not modify the contents pointed to by the source pointer. You can
prove this by modifying the function above, and its prototype, to
include the "const" modifier as shown. Then, within the function you can
add a statement which attempts to change the contents of that which is
pointed to by source, such as:

\begin{minted}{c}
    *source = 'X';
\end{minted}

which would normally change the first character of the string to an X.
The const modifier should cause your compiler to catch this as an error.
Try it and see.

Now, let's consider some of the things the above examples have shown us.
First off, consider the fact that \textbf{*ptr++} is to be interpreted
as returning the value pointed to by \textbf{ptr} and then incrementing
the pointer value. This has to do with the precedence of the operators.
Were we to write \textbf{(*ptr)++} we would increment, not the pointer,
but that which the pointer points to! i.e. if used on the first
character of the above example string the 'T' would be incremented to a
'U'. You can write some simple example code to illustrate this.

Recall again that a string is nothing more than an array of characters,
with the last character being a \textbf{'\textbackslash0'}. What we have
done above is deal with copying an array. It happens to be an array of
characters but the technique could be applied to an array of integers,
doubles, etc. In those cases, however, we would not be dealing with
strings and hence the end of the array would not be marked with a
special value like the nul character. We could implement a version that
relied on a special value to identify the end. For example, we could
copy an array of positive integers by marking the end with a negative
integer. On the other hand, it is more usual that when we write a
function to copy an array of items other than strings we pass the
function the number of items to be copied as well as the address of the
array, e.g. something like the following prototype might indicate:

\begin{minted}{c}
    void int_copy(int *ptrA, int *ptrB, int nbr);
\end{minted}

where \textbf{nbr} is the number of integers to be copied. You might
want to play with this idea and create an array of integers and see if
you can write the function \textbf{int\_copy()} and make it work.

This permits using functions to manipulate large arrays. For example, if
we have an array of 5000 integers that we want to manipulate with a
function, we need only pass to that function the address of the array
(and any auxiliary information such as nbr above, depending on what we
are doing). The array itself does \textbf{not} get passed, i.e. the
whole array is not copied and put on the stack before calling the
function, only its address is sent.

This is different from passing, say an integer, to a function. When we
pass an integer we make a copy of the integer, i.e. get its value and
put it on the stack. Within the function any manipulation of the value
passed can in no way effect the original integer. But, with arrays and
pointers we can pass the address of the variable and hence manipulate
the values of the original variables.

\begin{comment}
\href{ch4x.htm}{Chapter 4: More on Strings}

\href{pointers.htm}{Back to Table of Contents}
\end{comment}
