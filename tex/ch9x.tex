\hypertarget{chapter-9-pointers-and-dynamic-allocation-of-memory}{%
\subsection{CHAPTER 9: Pointers and Dynamic Allocation of
Memory}\label{chapter-9-pointers-and-dynamic-allocation-of-memory}}

There are times when it is convenient to allocate memory at run time
using \textbf{malloc()}, \textbf{calloc()}, or other allocation
functions. Using this approach permits postponing the decision on the
size of the memory block need to store an array, for example, until run
time. Or it permits using a section of memory for the storage of an
array of integers at one point in time, and then when that memory is no
longer needed it can be freed up for other uses, such as the storage of
an array of structures.

When memory is allocated, the allocating function (such as
\textbf{malloc()}, \textbf{calloc()}, etc.) returns a pointer. The type
of this pointer depends on whether you are using an older K\&R compiler
or the newer ANSI type compiler. With the older compiler the type of the
returned pointer is \textbf{char}, with the ANSI compiler it is
\textbf{void}.

If you are using an older compiler, and you want to allocate memory for
an array of integers you will have to cast the char pointer returned to
an integer pointer. For example, to allocate space for 10 integers we
might write:

\begin{minted}{c}
    int *iptr;
    iptr = (int *)malloc(10 * sizeof(int));
    if (iptr == NULL)

    { .. ERROR ROUTINE GOES HERE .. }
\end{minted}

If you are using an ANSI compliant compiler, \textbf{malloc()} returns a
\textbf{void} pointer and since a void pointer can be assigned to a
pointer variable of any object type, the \textbf{(int *)} cast shown
above is not needed. The array dimension can be determined at run time
and is not needed at compile time. That is, the \textbf{10} above could
be a variable read in from a data file or keyboard, or calculated based
on some need, at run time.

Because of the equivalence between array and pointer notation, once
\textbf{iptr} has been assigned as above, one can use the array
notation. For example, one could write:

\begin{minted}{c}
    int k;
    for (k = 0; k < 10; k++)
       iptr[k] = 2;
\end{minted}

to set the values of all elements to 2.

Even with a reasonably good understanding of pointers and arrays, one
place the newcomer to C is likely to stumble at first is in the dynamic
allocation of multi-dimensional arrays. In general, we would like to be
able to access elements of such arrays using array notation, not pointer
notation, wherever possible. Depending on the application we may or may
not know both dimensions at compile time. This leads to a variety of
ways to go about our task.

As we have seen, when dynamically allocating a one dimensional array its
dimension can be determined at run time. Now, when using dynamic
allocation of higher order arrays, we never need to know the first
dimension at compile time. Whether we need to know the higher dimensions
depends on how we go about writing the code. Here I will discuss various
methods of dynamically allocating room for 2 dimensional arrays of
integers.

First we will consider cases where the 2nd dimension is known at compile
time.

METHOD 1:

One way of dealing with the problem is through the use of the
\textbf{typedef} keyword. To allocate a 2 dimensional array of integers
recall that the following two notations result in the same object code
being generated:

\begin{minted}{c}

    multi[row][col] = 1;     *(*(multi + row) + col) = 1;
\end{minted}

It is also true that the following two notations generate the same code:

\begin{minted}{c}

    multi[row]            *(multi + row)
\end{minted}

Since the one on the right must evaluate to a pointer, the array
notation on the left must also evaluate to a pointer. In fact
\textbf{multi{[}0{]}} will return a pointer to the first integer in the
first row, \textbf{multi{[}1{]}} a pointer to the first integer of the
second row, etc. Actually, \textbf{multi{[}n{]}} evaluates to a pointer
to that array of integers that make up the n-th row of our 2 dimensional
array. That is, \textbf{multi} can be thought of as an array of arrays
and \textbf{multi{[}n{]}} as a pointer to the n-th array of this array
of arrays. Here the word \textbf{pointer} is being used to represent an
address value. While such usage is common in the literature, when
reading such statements one must be careful to distinguish between the
constant address of an array and a variable pointer which is a data
object in itself.

Consider now:

-----------  Program 9.1  -----------------------------------
\inputminted[breaklines]{c}{../src/ch9-1.c}
--------- end program 9.1 -----------------------------------

Here I have assumed an ANSI compiler so a cast on the void pointer
returned by \textbf{malloc()} is not required. If you are using an older
K\&R compiler you will have to cast using:

\begin{minted}{c}
    rptr = (RowArray *)malloc(.... etc.
\end{minted}

Using this approach, \textbf{rptr} has all the characteristics of an
array name name, (except that rptr is modifiable), and array notation
may be used throughout the rest of the program. That also means that if
you intend to write a function to modify the array contents, you must
use COLS as a part of the formal parameter in that function, just as we
did when discussing the passing of two dimensional arrays to a function.

METHOD 2:

In the METHOD 1 above, rptr turned out to be a pointer to type "one
dimensional array of COLS integers". It turns out that there is syntax
which can be used for this type without the need of \textbf{typedef}. If
we write:

\begin{minted}{c}

    int (*xptr)[COLS];
\end{minted}

the variable \textbf{xptr} will have all the same characteristics as the
variable \textbf{rptr} in METHOD 1 above, and we need not use the
\textbf{typedef} keyword. Here \textbf{xptr} is a pointer to an array of
integers and the size of that array is given by the \textbf{\#defined
COLS}. The parenthesis placement makes the pointer notation predominate,
even though the array notation has higher precedence. i.e. had we
written

\begin{minted}{c}
    int *xptr[COLS];
\end{minted}

we would have defined \textbf{xptr} as an array of pointers holding the
number of pointers equal to that \#defined by COLS. That is not the same
thing at all. However, arrays of pointers have their use in the dynamic
allocation of two dimensional arrays, as will be seen in the next 2
methods.

METHOD 3:

Consider the case where we do not know the number of elements in each
row at compile time, i.e. both the number of rows and number of columns
must be determined at run time. One way of doing this would be to create
an array of pointers to type \textbf{int} and then allocate space for
each row and point these pointers at each row. Consider:

-----------  Program 9.2  -----------------------------------
\inputminted[breaklines]{c}{../src/ch9-2.c}
--------- end program 9.2 -----------------------------------

In the above code \textbf{rowptr} is a pointer to pointer to type
\textbf{int}. In this case it points to the first element of an array of
pointers to type \textbf{int}. Consider the number of calls to
\textbf{malloc()}:

\begin{minted}{c}

    To get the array of pointers             1     call
    To get space for the rows                5     calls
                                          -----
                     Total                   6     calls
\end{minted}

If you choose to use this approach note that while you can use the array
notation to access individual elements of the array, e.g.
\textbf{rowptr{[}row{]}{[}col{]} = 17;}, it does not mean that the data
in the "two dimensional array" is contiguous in memory.

You can, however, use the array notation just as if it were a continuous
block of memory. For example, you can write:

\begin{minted}{c}
    rowptr[row][col] = 176;
\end{minted}

just as if rowptr were the name of a two dimensional array created at
compile time. Of course \textbf{row} and \textbf{col} must be within the
bounds of the array you have created, just as with an array created at
compile time.

If you want to have a contiguous block of memory dedicated to the
storage of the elements in the array you can do it as follows:

METHOD 4:

In this method we allocate a block of memory to hold the whole array
first. We then create an array of pointers to point to each row. Thus
even though the array of pointers is being used, the actual array in
memory is contiguous. The code looks like this:

-----------  Program 9.3  -----------------------------------
\inputminted[breaklines]{c}{../src/ch9-3.c}
--------- end program 9.3 -----------------------------------

Consider again, the number of calls to malloc()

\begin{minted}{c}
    To get room for the array itself      1      call
    To get room for the array of ptrs     1      call
                                        ----
                         Total            2      calls
\end{minted}

Now, each call to \textbf{malloc()} creates additional space overhead
since \textbf{malloc()} is generally implemented by the operating system
forming a linked list which contains data concerning the size of the
block. But, more importantly, with large arrays (several hundred rows)
keeping track of what needs to be freed when the time comes can be more
cumbersome. This, combined with the contiguousness of the data block
that permits initialization to all zeroes using \textbf{memset()} would
seem to make the second alternative the preferred one.

As a final example on multidimensional arrays we will illustrate the
dynamic allocation of a three dimensional array. This example will
illustrate one more thing to watch when doing this kind of allocation.
For reasons cited above we will use the approach outlined in alternative
two. Consider the following code:

-----------  Program 9.4  -----------------------------------
\inputminted[breaklines]{c}{../src/ch9-4.c}
--------- end program 9.4 -----------------------------------

If you have followed this tutorial up to this point you should have no
problem deciphering the above on the basis of the comments alone. There
are a couple of points that should be made however. Let's start with the
line which reads:

\begin{minted}{c}
    Arr3D[z][y] = space + (z*(X_DIM * Y_DIM) + y*X_DIM);
\end{minted}

Note that here \textbf{space} is a character pointer, which is the same
type as \textbf{Arr3D{[}z{]}{[}y{]}}. It is important that when adding
an integer, such as that obtained by evaluation of the expression
\textbf{(z*(X\_DIM * Y\_DIM) + y*X\_DIM)}, to a pointer, the result is a
new pointer value. And when assigning pointer values to pointer
variables the data types of the value and variable must match.

\begin{comment}
\href{ch10x.htm}{Chapter 10: Pointers to Functions}

\href{pointers.htm}{Back to Table of Contents}
\end{comment}
